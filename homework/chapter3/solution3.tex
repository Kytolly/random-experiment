\documentclass[12pt, a4paper, oneside]{ctexart}
\usepackage{amsmath, amsthm, amssymb, bm, color, framed, graphicx, hyperref, mathrsfs}

\title{\textbf{《应用随机过程》第三次作业}} 
\author{24 2022010910017 谢卿云}
\date{\today}
\linespread{1.5}
\definecolor{shadecolor}{RGB}{241, 241, 255}
\newcounter{problemname}
\newcounter{lemmaname}
\newenvironment{problem}{\begin{shaded}\stepcounter{problemname}\par\noindent\textbf{题目\arabic{problemname}. }}{\end{shaded}\par}
\newenvironment{lemma}{\stepcounter{lemmamname}\par\noindent\textbf{引理\arabic{problemname}. }}{\par}
\newenvironment{solution}{\par\noindent\textbf{解答. }}{\par}
\newenvironment{note}{\par\noindent\textbf{题目\arabic{problemname}的注记. }}{\par}

\begin{document}
	
	\maketitle
	
	\begin{problem}
		(泊松过程的数字特征)设 $\{N(t), t \geq 0\}$ 是强度为 $\lambda$ 的泊松过程。求: $N(t)$ 的均值函数、方差函数、协方差函数、和自相关函数。
	\end{problem}
	
	\begin{solution}
		回忆Poisson过程的定义,$N(t)\sim P(\lambda t)$.不妨假设$s\le t$.
		$$\begin{aligned}
			&\mu_N(t)=E[N(t)]=\lambda t\\
			&D_N(t)=D[N(t)]=\lambda t\\
		\end{aligned}$$
		$$\begin{aligned}
			&R_N(s,t)=E[N(s)N(t)]=E[N(s)(N(s)+N(t)-N(s))]\\
			&=E[N^2(s)]+E[N(s)]E[N(t)-N((s))]\\
			&=\lambda s+\lambda^2 s^2+ \lambda s(\lambda(t-s))\\
			&=\lambda \min\{s,t\}+\lambda^2st\\
			&\gamma_N(s,t)=R_N(s,t)-E[N(s)]E[N(t)]\\
			&=\lambda \min\{s,t\}+\lambda^2st-\lambda^2st=\lambda \min\{s,t\}
		\end{aligned}$$
	\end{solution}
	
%	\begin{note}
%		这里是注记. 
%	\end{note}
	
	\begin{problem}
		设某电话总机在 $t$ 分钟内接到的电话呼叫数 $\{N(t), t \geq 0\}$ 是具有速率为 $\lambda$ 的Poisson过程,求:\\
		(a) 3 分钟内接到 5 次呼叫的概率 $p_{1}$.\\
		(b) 已知 3 分钟内接到 5 次呼叫,问第 5 次呼叫在第 2 到第 3 分钟之间到来的概率 $p_{2}$.
	\end{problem}
	
	\begin{solution}
		\\ (a)注意到$N(t)\sim P(\lambda t)$.由Poisson分布律\\
		$$p_1=P(\{N(3)=5\})=\frac{(3\lambda)^5}{5!}e^{-3\lambda}$$
		\\ (b)
		记$T_n$为电话呼叫事件第$n$次发生的时刻.\\
		$$\begin{aligned}
			&p_2=1-P({N(2)=5|N(3)=5})\\
			&=1-\frac{P({N(2)=5})P({N(3)-N(2)=0})}{P(N(3)=5)}\\
			&=1-\frac{\frac{(2\lambda)^5}{5!}e^{-2\lambda} \lambda^0 e^{-\lambda}}{\frac{(3\lambda)^5}{5!}e^{-3\lambda} }\\
			&=1-(\frac 2 3)^5
		\end{aligned}$$
		
%		$$p_2=P({2\le T_5<3|N(3)=5})   =P({N(2)<5}|N(3)=5)
%		   =P(\bigcup_{k=0}^{4}\{N(2)=k|N(3)=5\})   =\sum_{k=0}^4 \frac{2\lambda}{k!}e^{-2\lambda}$$
%		$$p_2=1-P({N(2)=5|N(3)=5})=1-\frac{P({N(2)=5})P({N(3)-N(2)=0})}{P(N(3)=5)}$$
%		$$=1-\frac{\frac{(2\lambda)^5}{5!}e^{-2\lambda} \lambda^0 e^{-\lambda}}{\frac{(3\lambda)^5}{5!}e^{-3\lambda} }=1-(\frac 2 3)^5$$		
	\end{solution}
	
	\begin{problem}
		设 $\{N(t), t \geq 0\}$ 是参数为 $\lambda$ 的Poisson过程, 求:\\
		(a) $s>0$ 时, $E[N(s) N(t+s)]$;\\
		(b) $0<s<t$ 时, $P(N(s)=k \mid N(t)=n)$;\\
		(c) $s>0$ 时, $P(N(t+s)=j \mid N(s)=j)$.\\
	\end{problem}
	
	\begin{solution}
		\\ (a)由泊松过程的独立增量性质.
		$$\begin{aligned}
				&E[N(s)N(t+s)]=E[N(s)(N(t+s)-N(s)+N(s))]\\
						&=E[N^2(s)]+E[N(s)]E[N(t+s)-N(s)]\\
			 			&=E^2[N(s)]+D(N(s))+E[N(s)]E[N(t+s)-N(s)]\\
						&=\lambda^2s^2+\lambda s+\lambda^2 st\\
		\end{aligned}$$
		(b)假设$k\le n$.(否则为平凡情况,概率为0).利用泊松过程的独立增量性质.
		$$\begin{aligned}
			&P(N(s)=k \mid N(t)=n)\\
			&=\frac{P({N(s)=k,N(t)=n})}{P({N(t)=n})}\\
			&=\frac{\frac{(\lambda s)^k}{k!}e^{-\lambda s}  \frac{(\lambda (t-s))^{n-k}}{(n-k)!}e^{-\lambda (t-s)}}{\frac{(\lambda t)^n}{n!}e^{-\lambda t}}\\
			&=\binom{n}{k}\left(\frac s t\right)^k\left(\frac{t-s}t\right)^{n-k}\\
		\end{aligned}$$
		\\ (c) 由泊松过程的独立增量性质,
		$$\begin{aligned}
			&P(N(t+s)=j \mid N(s)=j)\\
		&=P({N(t+s)-N(s)=0\mid N(s)=j})\\
		&=P({N(t+s)-N(s)=0})=e^{-\lambda t}\\	
		\end{aligned}$$
		
	\end{solution}
	\begin{note}
		我们在(b)中实际证明引理1:已知$s<t,k\le n,N(t)=n$,则$N(s)\sim B(n,\frac s t)$ 
	\end{note}
	
	\begin{problem}
		设 $\left\{N_{1}(t), t \geq 0\right\}$ 和 $\left\{N_{2}(t), t \geq 0\right\}$ 分别是参数为 $\lambda_{1}$ 和 $\lambda_{2}$ 的泊松过程, 且这两个过程相互独立。对 $0 \leq k \leq n$, 证明下列成立:\\
		(a) $P\left(N_{1}(t)=k \mid N_{1}(t)+N_{2}(t)=n\right)=C_{n}^{k}\left(\frac{\lambda_{1}}{\lambda_{1}+\lambda_{2}}\right)^{k}\left(\frac{\lambda_{2}}{\lambda_{1}+\lambda_{2}}\right)^{n-k}$;\\		
		(b) $E\left[N_{1}(t) \mid N_{1}(t)+N_{2}(t)=n\right]=\frac{n \lambda_{1}}{\lambda_{1}+\lambda_{2}}$;
		
	\end{problem}
	
	\begin{solution}
		\\ (a)不难根据全概率公式得出
		$$\begin{aligned}
			&P\left(\{ N_{1}(t)=k \mid N_{1}(t)+N_{2}(t)=n\right)=k \mid N(t)=n\})\\
			&=\frac{P({N_{1}(t)=k,N_{2}(t)=n-k})}{P({N_{1}(t)+N_{2}(t)=n})}\\
			&=\frac{P({N_1(t)=k})P({N(2)=n-k})}{\sum_{k=0}^{n}P({N_1(t)=k})P({N(2)=n-k})}\\
			&=\frac{ \frac{{\lambda_1 t}^{k}}{k!}e^{-\lambda_1 t} \frac{{\lambda_2 t}^{n-k}}{(n-k)!}e^{-\lambda_2 t} }{\sum_{k=0}^{n} \frac{{\lambda_1 t}^{k}}{k!}e^{-\lambda_1 t} \frac{{\lambda_2 t}^{n-k}}{(n-k)!}e^{-\lambda_2 t} }\\
			&=\binom n k \left(\frac{\lambda_1}{\lambda_1+\lambda_2}\right)^k\left(\frac{\lambda_2}{\lambda_1+\lambda_2}\right)^{n-k}\\
		\end{aligned}$$
		\\ (b)
			$$\begin{aligned}
			&E\left[N_{1}(t) \mid N_{1}(t)+N_{2}(t)\right]\\
			&=\sum_{k=0}^n kP\left(N_{1}(t)=k \mid N_{1}(t)+N_{2}(t)=n\right)\\
			&=\sum_{k=0}^n k \binom n k \left(\frac{\lambda_1}{\lambda_1+\lambda_2}\right)^k\left(\frac{\lambda_2}{\lambda_1+\lambda_2}\right)^{n-k}\\
			&=\frac{n\lambda_1}{\lambda_1+\lambda_2}\sum_{k=0}^n  \binom{n-1}{k-1}  \left(\frac{\lambda_1}{\lambda_1+\lambda_2}\right)^{k-1}\left(\frac{\lambda_2}{\lambda_1+\lambda_2}\right)^{n-k}\\
			&=\frac{n\lambda_1}{\lambda_1+\lambda_2}\left(\frac{\lambda_1}{\lambda_1+\lambda_2} + \frac{\lambda_1}{\lambda_1+\lambda_2} \right)^{n-1}\\
			&=\frac{n\lambda_1}{\lambda_1+\lambda_2}\\
		\end{aligned}$$
		
		
		
	\end{solution}
	
	
	
	\begin{problem}
		假设 $[0, t]$ 内顾客到达商场的人数 $\{N(t), t \geq 0\}$ 是强度为 $\lambda$ 的Poisson过程, 且每一个到达商场的顾客是男性还是女性的概率分别为 $p$ 和 $q, p+q=1$. 设 $N_{1}(t)$ 和 $N_{2}(t)$ 分别为 $[0, t]$ 内到达商场的男女顾客数。求 $N_{1}(t)$ 和 $N_{2}(t)$ 的分布, 并证明它们相互独立.
	\end{problem}
	
	\begin{solution}
	断言:\\
	$$N_1(t)\sim P(p\lambda),N(2)\sim P(q\lambda),N_1(t),N_2(t)\text{相互独立}$$
	为此只需证明以下Poisson过程分流引理:
	设$\{N(t)|t\geq 0\} \sim P(\lambda)$,$\{Y_j\}_{j\geq 0}\sim B(1,p)$,相互独立且与$N(t)$独立,令\\
	$$N_1(t)=\sum_{k=0}^{N(t)}Y_j,N_1(t)=\sum_{k=0}^{N(t)}(1-Y_j)$$则$N_1(t)\sim P(\lambda p),N(2)\sim P(\lambda(1-p)),N_1(t),N_2(t)\text{相互独立}$\\
	对于$0\le s <t$,$$M(t)=\sum_{j=1}^{N(t)}Y_j,t\ge 0$$
	是$(s,t]$内事件的到达数,仿照引理1,我们可知已知$N(s)=l,N(t)=k$的条件下$M(t)-M(s)\sim B(k-l,p)$,也即$E[I_{M(t)-M(s)=n}|N(s),N(t)]=g(N(t)-N(s),n)$
	两边取期望得$P({M(t)-M(s)=n})=E[g(N(t)-N(s),n)]$,由${N(t)}$的平稳性可知$M(t)$为平稳增量过程.\\
	再来证明独立增量性,对正整数$m$和$0=t_0\le t_1<...< t_m$和整数$0=n_0\le n_1\le n_2 \le ...\le n_m$,定义$$\mathbf{N}=(N(t_1),...,N(t_m)),\mathbf{n}=(n_1,...,n_m)$$
	注意到独立性:\\
	$$\begin{aligned}
		&P(M(t_j)-M(t_{j-1})=k_j, 1\leq j\leq m\mid\boldsymbol{N}=\boldsymbol{n}) \\
		=& P\left(\sum_{i=n_{j-1}}^{n_j}Y_i=k_j, 1\leq j\leq m\mid\boldsymbol{N}=\boldsymbol{n}\right) \\
		=& P\left(\sum_{i=n_{j-1}}^{n_{j}}Y_{i}=k_{j},~1\leq j\leq m\right) \\
		=& \prod_{j=1}^mP\left(\sum_{i=n_{j-1}}^{n_j}Y_i=k_j\right) \\
		=&\prod_{j=1}^mg(n_j-n_{j-1},k_j).
	\end{aligned}$$
	于是\\
	$$E[I_{\{M(t_j)-M(t_{j-1})=k_j\}}|\mathbf{N}]=\prod_{j=1}^mg(N(t_j)-N(t_{j-1}),k_j)$$
	独立增量性得证,只需验证$M(t)$服从参数为$p\lambda t$的Poisson分布:
	$$\begin{aligned}
		&P(M(t)=n) \\
		=& E[g(N(t),n)] \\
		=& \sum_{k=n}^\infty g(k,n)P(N(t)=k) \\
		=&\sum_{k=n}^\infty\binom knp^nq^{k-n}\frac{\lambda^k}{k!}e^{-\lambda} \\
		=& \frac{(p\lambda)^n}{n!}e^{-p\lambda}, n=0,1,\ldots. 
	\end{aligned}$$
	显然有$M(t)=N_1(t)$,同理对$N_2(t)$也成立,最后一步就要验证两过程的独立性:
	对正整数$n$和$0=t_0\le t_1<...< t_n,0=k_0\le k_1<...< k_n,0=m_0\le m_1<...< m_n$,记$n_j=k_j+m_j,\mathbf{n}=(n_1,...,n_n)$,随机变量\\
	$$
	\xi_j=\sum_{i=n_{j-1}+1}^{n_j}Y_i,j=1,2,...
	$$
	服从二项分布,相互独立,且与$\{N(t)\}$独立,于是\\
	$$\begin{aligned}
		&P(N_{1}(t_{j})=k_{j},N_{2}(t_{j})=m_{j},1\leq j\leq n) \\
		=& P(N_1(t_j)=k_j,N(t_j)=n_j,1\leq j\leq n) \\
		=& P(N_1(t_j)-N_1(t_{j-1})=k_j-k_{j-1},N(t_j)=n_j,1\leq j\leq n) \\
		=& P(\xi_j=k_j-k_{j-1},N(t_j)-N(t_{j-1})=n_j-n_{j-1},1\leq j\leq n) \\
		=& P(\xi_j=k_j-k_{j-1},1\leq j\leq n) \\
		&P(N(t_j)-N(t_{j-1})=n_j-n_{j-1},1\leq j\leq n) \\
		=& \prod_{j=1}^{n}P(\xi_{j}=k_{j}-k_{j-1})P(N(t_{j})-N(t_{j-1})=n_{j}-n_{j-1}) \\
		=&\prod_{j=1}^n\frac{(n_j-n_{j-1})!}{(k_j-k_{j-1})!(m_j-m_{j-1})!}p^{k_j-k_{j-1}}q^{m_j-m_{j-1}} 
		\cdot\frac{[\lambda(t_j-t_{j-1})]^{n_j-n_{j-1}}}{(n_j-n_{j-1})!}e^{-\lambda(t_j-t_{j-1})} \\
		=&\prod_{j=1}^n\frac{[p\lambda(t_j-t_{j-1})]^{k_j-k_{j-1}}}{(k_j-k_{j-1})!} \\
		=&\prod_{j=1}^n\frac{[q\lambda(t_j-t_{j-1})]^{m_j-m_{j-1}}}{(m_j-m_{j-1})!}.
	\end{aligned}$$
	因此两过程独立.
	\end{solution}
\begin{note}
	用归纳法不难推广到分解成多个相互独立的Poisson过程.
\end{note}

	\begin{problem}
		设某个汽车站有 $A, B$ 两辆跑同一路线的长途汽车, 设到达该站的旅客数为一Poisson过程, 平均每 10 分钟到达 15 位旅客, 而每个旅客进入 $A$ 或 $B$ 的概率分别为 $\frac 2  3,\frac 1  3$. 设 $N_{A}(t)$, $N_{B}(t)$ 分别表示时段 $[0, t]$ 内进入 $A$ 或 $B$ 的旅客数, 求:\\
		(a) $N_{A}(t), N_{B}(t)$ 的分布。\\		
		(b) 若 $A$ 车旅客数达到 10 位即发车, $B$ 车旅客数达到 15 位即发车, 求 $A, B$ 的等待时间的分布, 并求 $A$ 比 $B$ 先开车的概率。(提示:若 $X \sim \operatorname{Gamma}(\alpha, \beta)$ 且 $\alpha$ 为正整数, 则
		
		$$
		F_{X}(x)=\int_{0}^{x} f_{X}(x) d x=1-\sum_{i=0}^{\alpha-1} \frac{(\beta x)^{i}}{i !} e^{-\beta x}
		$$
		
		最终的概率无需算出具体值。)
	\end{problem}
	
	\begin{solution}
		定义随机变量$X_i=\begin{cases}
			1&\text{,第i名乘客进入A车}\\
			0&\text{,第i名乘客进入B车}\\
		\end{cases},Y_i=1-X_i,i=0,1,...$\\
		(1)注意到$N_A(t)+N_B(t)\sim P(\frac 3 2 t)$,根据泊松过程分流引理,$\{N_A(t)|t\ge 0\},\{N_B(t)|t\ge 0 \}$均为泊松过程,参数分别为$\frac 2 3\times \frac 3 2 = 1, \frac 1 3\times \frac 3 2=\frac 1 2 $ ,即$N_A(t)\sim P(1),N_B(t)\sim P(0.5)$\\
		(2)取$T_A(n),T_B(n)$为第$n$个旅客进入A,B车的时刻.则根据泊松过程相关分布的性质,$T_A(n)\sim \Gamma(n,1),T_B(n)\sim \Gamma(n,0.5)$\\
		那么A的等待时间服从$\Gamma(10,1)$,B车的等待时间服从$\Gamma(15,0,5)$,A比B先开车的概率为
		$$P({T_A(10)\le T_B(15)})=\int_{-\infty}^{+\infty}P({T_A(10)\le x\mid T_B(15)=x} )dP({T_B(15)=x} ) $$
		$$ =\int_{x}^{+\infty}(1-\sum_{i=0}^{9} \frac{x^i}{i!}e^{-x}) \frac{1}{2^n\Gamma(15)}x^14e^{-\frac 1 2 x}dx$$
		
	\end{solution}
	
	
	\begin{problem}
		设某医院专家门诊从早上 $8: 00$ 开始就已有无数患者等候, 而每次专家只能为一名患者服务,服务的平均时间为 20 分钟,且每名患者接受服务的时间服从独立的指数分布。求从8:00到12:00门诊结束时接受过治疗的患者在医院停留的平均时间。
	\end{problem}
	
	\begin{solution}
		默认8:00为0时刻,记$T_n$表示第$n$位患者离开的时刻,$S_n$表示第$n$位患者接收服务的时长,由题意,$S_n\sim Exp(\frac 1 {20})$,注意到,$T_n=\sum_{i=1}^{n}S_i,T_n\sim \Gamma(n,\frac 1 {20} )$,患者的离开是一个Poisson过程,换言之,记$(0,t]$内离开的患者数为$N(t)$,$N(t)\sim P(\frac 1 {20})$.
		注意到$\sum_{i=1}^{n}T_i\sim \Gamma(\frac{n(n+1)}{2},\frac 1 {20}) \Gamma E(\frac{\sum_{i=1}^{n}T_i}{n})$
		由全期望公式和$\Gamma$分布的性质:
		$$
		E[
		\frac{
			\sum_{i=1}^{N(240)}T_i
			}{N(240)}
		]=
		E[
			E(\frac{\sum_{i=1}^{N(240)}T_i}{N(240)} )
			|N(240)
		]$$
				$$ 		=\sum_{i=1}^{\infty} E( \frac{\sum_{i=1}^{N(240)}T_i}{N(240)} \mid {N(240)=n} ) P({N(240)=n})$$
					$$ = \sum_{i=1}^{\infty}  \frac{\frac1 2 n(n+1)   \cdot20}{n} (\frac{240\cdot \frac{1}{20} }{n!} e^{-240\cdot\frac1{20}})) =\frac{130e^{12}-10}{e^{12}}$$
	\end{solution}
	
	\begin{problem}
		设每天经过某路口的车辆数为: 早上 $7: 00$ 到 $8: 00,11: 00$ 到 $12: 00$ 为平均每分钟 2 辆, 其他时间平均每分钟 1 辆。则早上 $7: 30$ 到中午 $11: 20$ 平均有多少辆汽车经过此路口? 这段时间经过此路口的车辆数超过 500 辆的概率是多少?
		
	\end{problem}
	
	\begin{solution}
		我们默认7:00为0时刻,记$(0,t]$内路口经过的车辆数为$N(t)$,容易写出强度函数$\lambda(t)=\begin{cases}
			2&,t\in [0,60]\\
			1&,t\in (60,240]\\
			2&,t\in (2400,300]
		\end{cases}$
		注意到$\int_{30}^{260}\lambda(t)dt= 30\times 2+180\times 1+2\times 20=280$,则$N(260)-N(30)\sim P(280)$\\
		则早上 $7: 30$ 到中午 $11: 20$ 平均有$E[N(260)-N(30)]=280$辆车经过,超过500辆的概率为$P({N(260)-N(30)>500})=\sum_{k=501}^{+\infty}\frac{280^n}{n!}e^{-280}$\\
	\end{solution}
	
	\begin{problem}
		设 $\{N(t), t \geq 0\}$ 是强度函数为 $\lambda(t)$ 的非齐次Poisson过程, $X_{1}, X_{2}, \cdots$ 是事件之间的时间间隔,问:\\
		(a) $X_{i}$ 之间是否相互独立?\\		
		(b) $X_{i}$ 之间是否同分布?(提示:求 $X_{1}$ 和 $X_{2}$ 的分布)
	\end{problem}
		
	\begin{solution}
		 记$m(t)=\int_{0}^{t}\lambda(s)ds$.\\
		 (1) $P({X_2>t|X_1=u})=P({N(t)-N(u)=0\mid X_1=s})=P({N(t)-N(u)=0})=e^{\int_{u}^{t+u}\lambda(s)ds}=e^{m(u+t)-m(u)}$与$u$有关,从而${X_i}$并不相互独立.\\
		 (2) 注意到$P(X_1>t)=e^{-m(t)}$\\
		 $$P(X_2>t)=\int_{0}^{\infty}P({X_2>t\mid X_1=u})dP({X_1=u})$$
		 $$= \int_{0}^{\infty} e^{m(u+t)-m(u)} \lambda(u)e^{-m(u)}du
		  = \int_{0}^{\infty} e^{m(u+t)} \lambda(u)du$$
		 因此${X_i}$并不同分布.
	\end{solution}
	
	\begin{problem}
		 设 $\left\{N_{1}(t), t \geq 0\right\}$ 和 $\left\{N_{2}(t), t \geq 0\right\}$ 分别是参数为 $\lambda_{1}$ 和 $\lambda_{2}$ 的泊松过程, 且这两个过程相互独立。问: $\left\{N_{1}(t)-N_{2}(t), t \geq 0\right\}$ 是否为复合Poisson过程?
	\end{problem}
	
	\begin{solution}
		结论是肯定的,只需考察特征函数:令$X(t)=N_1(t)-N_2(t)$
		$$\begin{aligned}
			&\psi_{X(t)}(u)=E[e^{-iuX(t)}]=E[e^{-iuN_1(t)}]E[e^{iuN_2(t)}]\\
			&=\sum_{n=0}^{\infty}\frac{(\lambda_1te^{-iu})^n}{n!}e^{-\lambda_1t}		\sum_{n=0}^{\infty}\frac{(\lambda_2te^{iu})^n}{n!}e^{-\lambda_2t}\\
			&= e^{\lambda_1t(e^{-iu}-1)}  e^{\lambda_2t(e^{iu}-1)}\\
			&=e^{(\lambda_1+\lambda_2)t\left(\frac{\lambda_1}{\lambda_1+\lambda_2}e^{-iu}+\frac{\lambda_2}{\lambda_1+\lambda_2}e^{iu}-1\right)}\\
		\end{aligned}$$
		由特征函数和分布唯一确定定理,$\left\{N_{1}(t)-N_{2}(t), t \geq 0\right\}$ 为复合Poisson过程.
	\end{solution}
	
	
	\begin{problem}
		设某飞机场到达的客机数是一个Poisson过程, 平均每小时到达 10 架。客机共有三种类型, 能承载的乘客数分别为 200 人, 150 人, 100 人, 且三种飞机出现的概率为 $1 / 6,1 / 2,1 / 3$ 。令 $X$ 表示 5 小时内到达该机场的乘客数,求 $X$ 的期望和方差。
	\end{problem}
	
	\begin{solution}
		根据泊松过程分流引理,三种飞机在$(0,t]$的到达的数量$N_1(t),N_2(t),N_3(t) $均为泊松过程,换言之,$N_1(t)\sim P(\frac{10}{6}) ,N_2(t)\sim P(\frac{10}{2}),N_3(t)\sim P(\frac{10}{3}) $\\
		$$E(X)=E[200N_1(5)+150N_2(5)+100N_3(5)]$$$$=200E[N_1(5)]+150E[N_2(5)]+100E[N_3(5)]=\frac{3350}{3}$$
		$$D(X)=D[200N_1(5)+150N_2(5)+100N_3(5)]$$$$=200^2D[N_1(5)]+150^2D[N_2(5)]+100^2D[N_3(5)]=182500$$
	\end{solution}
	
	
\end{document}
